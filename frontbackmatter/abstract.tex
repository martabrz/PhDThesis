%*******************************************************
% Abstract
%*******************************************************
%\renewcommand{\abstractname}{Abstract}
\pdfbookmark[1]{Abstract}{Abstract}
\begingroup
\let\clearpage\relax
\let\cleardoublepage\relax
\let\cleardoublepage\relax

\chapter*{Abstract}
This thesis covers selected recent developments in the field of topological aspects of condensed matter physics. In particular, we focus on three directions which can be seen as extensions of well-established classification of free-fermionic gapped states: i) investigating a realization of topological states in the systems defined in non-integer spatial dimensions, ii) the role of crystal symmetries and how they affect the distinction between topologically trivial and non-trivial states, and iii) non-Hermitian Hamiltonians arising from a minimal modelling of gains and losses exhibiting observable phenomena without Hermitian counterparts. In all cases, we propose material candidates or experimental setups to support our theoretical findings.
\endgroup

\cleardoublepage%

\begingroup
\let\clearpage\relax
\let\cleardoublepage\relax
\let\cleardoublepage\relax

\begin{otherlanguage}{polish}
\pdfbookmark[1]{Abstrakt}{Abstrakt}
\chapter*{Abstrakt}
Przedłożona rozprawa doktorska obejmuje wybrane najnowsze osiągnięcia w dziedzinie topologicznych aspektów fizyki materii skondensowanej. W szczególności uwaga zostanie poświęcona trzem kierunkom badań, które można postrzegać jako rozszerzenie ugruntowanej klasyfikacji układów nieoddziałujących fermionów: i) próba realizacji stanów topologicznych w układach scharakteryzowanych przez liczbę wymiarów przestrzennych będących liczbą niecałkowitą, ii) rola symetrii krystalicznych i ich wpływ na rozróżnienie pomiędzy stanami topologicznymi oraz trywialnymi, a także iii) hamiltonianami niehermitowskimi będącymi efektywnym opisem układów otwartych i wykazującymi obserwowalne zjawiska bez odpowiedników w modelach hermitowskich. We wszystkich omawianych zagadnieniach zostaje przedyskutowana możliwa realizacja eksperymentalna w celu poparcia wyników teoretycznych.
\end{otherlanguage}

\endgroup

\vfill