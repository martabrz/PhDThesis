%*******************************************************
% Abstract
%*******************************************************
%\renewcommand{\abstractname}{Abstract}
\pdfbookmark[1]{Abstract}{Abstract}
\begingroup
\let\clearpage\relax
\let\cleardoublepage\relax
\let\cleardoublepage\relax

\chapter*{Abstract}
Topological states of matter have arisen as one of the rapidly growing fields in condensed matter physics, holding promise for future technologies, including spintronics and quantum computing. The notion of topology underlies exotic phenomena such as emergent collective excitations or protected edge modes exhibited in the low-energy spectrum. In particular, in non-interacting fermionic systems the connection between bulk and boundary degrees of freedom is established by the celebrated \emph{bulk-boundary correspondence}. In the presence (or absence) of fundamental symmetries: time-reversal, particle-hole and chiral, it is possible to classify free-fermionic, gapped systems in different spatial dimensions by the means of the ten-fold way. This classification scheme of topological insulators and superconductors tabulates all possible combinations of aforementioned symmetries and assigns a relevant topological invariant to each symmetry class. Despite its immense success, the ten-fold way turned out to be incomplete in the light of recent theoretical developments and experimental efforts.

This thesis discusses topological aspects of selected free-fermionic systems in low dimensions ($d \leq 2$) that go \emph{beyond} the ten-fold way. We address three distinct research directions in which the existing classification can be extended. In the first part, we investigate the Hofstadter model on two fractal geometries, namely the Sierpiński carpet and gasket. While being embedded in two-dimensional space, these fractals are characterized by a non-integer Hausdorff dimension. In addition, their connectivity properties are in a stark contrast to regular lattices as there is no clear distinction between edges and bulk. We numerically study the spectral and eigenstates localization properties, and observe a hierarchy of edge-like states located at different fractal depths. We employ topological invariants defined in real space: the Bott index and the Chern number, and identify regions in the energy spectrum with non-trivial topology. We further compute the phase diagram in the presence of disorder and conclude that characteristic features of the integer quantum Hall effects are also observed in \emph{almost} two-dimensional systems.

The second part of this dissertation is devoted to the significance of spatial symmetries. We start with concrete examples of group-V monolayers: atomically thin layers of bismuth and antimony, described within a tight-binding approximation. We show that a free-standing layer of bismuth hosts a quantum spin Hall phase, whereas a single layer of antimony has a trivial $\mathbb{Z}_2$ invariant. Applying a moderate strain to free-standing buckled layers, however, results in completely flat structures called bismuthene and antimonene, which are realizing a topological crystalline insulating phase protected by the mirror symmetry along the $z$ axis. Apart from the direct computations of relevant topological invariants, we use entanglement measures as complementary tools to define the bulk topology. We present how the full spectrum of the reduced density matrix corresponding to a spatially separated subsystem allows to differentiate between distinct topological states. Additionally, we study the scaling of the entanglement entropy across different topological phase transitions driven by doping, external electric field and strain.
An even more profound consequence of the crystal symmetries is the existence of \emph{obstructed atomic limits}, \ie systems for which the strong topological indices are trivial, but are not adiabatically connected to a trivial atomic limit. We propose a classification scheme for obstructed atomic limits in two dimensions, where Wilson loops and symmetry indicators play the role of topological invariants. We find that a buckled monolayer of antimony, among other suggested material candidates, is actually an obstructed limit and exhibits symmetry-protected corner charges.

In the third part, we discuss the interplay between topology and non-Hermiticity in Hamiltonians providing an effective description of open systems. Introducing non-Hermiticity leads to unique features such as exceptional points or the anomalous localization of all eigenstates at the boundary (the so-called skin effect). Using the $\pi$-flux model on a square lattice with a non-Hermitian extension, we demonstrate a novel phenomenon dubbed the \emph{reciprocal} skin effect, which does not require any direction-dependent hoppings. Theoretical predictions are supported by experimental results obtained by measuring a topolectrical circuit, which realizes the desirable physics of the non-Hermitian $\pi$-flux model.

\vspace{1cm}
\hspace{-0.4cm}\textbf{Keywords:} topological phases, tight-binding models, fractals, the Hofstadter model, entanglement, topological (crystalline) insulators, non-Hermitian Hamiltonians
\endgroup

\cleardoublepage%

\begingroup
\let\clearpage\relax
\let\cleardoublepage\relax
\let\cleardoublepage\relax

\begin{otherlanguage}{polish}
\pdfbookmark[1]{Streszczenie}{Streszczenie}
\chapter*{Streszczenie}
Topologiczne fazy materii stały się jedną z niezwykle szybko rozwijających się dziedzin fizyki materii skondensowanej ze względu na obiecujące przyszłościowe zastosowania na polu spintroniki oraz informatyki kwantowej. Pojęcie topologii leży u podstaw egzotycznych zjawisk takich jak kolektywne wzbudzenia lub chronione stany brzegowe występujące w niskoenergetycznej części widma. W szczególności, dla układów nieoddziałujących fermionów istnieje \emph{zasada korespondencji} między stopniami swobody materiału objętościowego a jego brzegu. W obecności (lub przy braku) wewnętrznych symetrii: odwrócenia w~czasie, typu cząstka-dziura i chiralnej, możliwa jest klasyfikacja układów swobodnych fermionów posiadających przerwę w widmie energetycznym materiału objętościowego zdefiniowanych w różnych wymiarach przestrzennych. Zestawienie wszystkich możliwych kombinacji wyżej wymienionych symetrii prowadzi do wyróżnienia dziesięciu klas symetrii i~przypisania im odpowiednich niezmienników topologicznych. Pomimo ogromnego sukcesu tej klasyfikacji izolatorów i nadprzewodników topologicznych, okazała się ona niewystarczająca w świetle ostatnich dokonań teoretycznych i eksperymentalnych.

Przedłożona rozprawa doktorska dotyczy efektów topologicznych w wybranych nieoddziałujących układach elektronowych, które \emph{nie} należą do jednej z dziesięciu wspomnianych klas symetrii. Przedstawione zostaną trzy odrębne kierunki badań, na bazie których można rozszerzyć istniejącą klasyfikację. W pierwszej części analizowany jest model Hofstadtera zdefiniowany na dwóch geometriach fraktalnych: dywanie i~trójkącie Sierpińskiego. Fraktale te, osadzone w przestrzeni dwuwymiarowej, charakteryzują się niecałkowitym wymiarem Hausdorffa. Istotną różnicą w porównaniu do sieci regularnych jest brak możliwości określenia które węzły sieci stanowią materiał objętościowy, a które należą do brzegu. Numerycznie przestudiowane zostały własności spektralne i lokalizacja stanów własnych, gdzie zaobserowano hierarchię stanów krawędziowych usytuowanych na różnych poziomach głębokości fraktalnej. Przy pomocy liczby Cherna w przestrzeni rzeczywistej oraz indeksu Botta, zidentyfikowano regiony w widmie energetycznym, w których stany mają nietrywialną topologię. Następnie przeanalizowano diagram fazowy w funkcji nieporządku i wywioskowano, że charakterystyczne cechy całkowitego kwantowego efektu Halla są również obserwowane w układach \emph{prawie} dwuwymiarowych.

Druga część rozprawy poświęcona jest znaczeniu symetrii wynikających ze struktury przestrzennej sieci krystalicznych. Konkretnymi przykładami są tutaj monowarstwy grupy V: atomowo cienkie warstwy bizmutu i antymonu, opisane w ramach modeli ciasnego wiązania. Wykazano, że wolnostojąca warstwa bizmutu wykazuje kwantowy spinowy efekt Halla, podczas gdy pojedyncza warstwa antymonu jest opisana trywialnym niezmiennikiem $\mathbb{Z}_2$. Naprężanie tych układów prowadzi do otrzymania całkowicie płaskich warstw zwanych bizmutenem i antymonenem, które realizują fazę topologicznego izolatora krystalicznego chronionego przez symetrię lustrzaną wzdłuż osi $z$. Oprócz bezpośrednich obliczeń odpowiednich niezmienników topologicznych, zostały zastosowane miary splątania jako uzupełniające narzędzia do zdefiniowania topologii. Zaprezentowano w jaki sposób pełne widmo zredukowanej macierzy gęstości, odpowiadające przestrzennie oddzielonemu podsystemowi, pozwala na rozróżnienie między różnymi stanami topologicznymi. Dodatkowo zbadano skalowanie entropii splątania podczas topologicznych przejść fazowych spowodowanych przez domieszkowanie, zewnętrzne pole elektryczne oraz naprężenie. 
Jeszcze bardziej fundamentalną konsekwencją symetrii krystalicznych jest istnienie \emph{ograniczonych izolatorów atomowych} (ang. \emph{obstructed atomic limits}), tj. układów dla których silne indeksy topologiczne (jak liczba Cherna) są trywialne, ale nie są topologicznie równoważne izolatorom trywialnym w których stany elektronowe lokalizują się na węzłach sieci krystalicznej. Została zaproponowana klasyfikacja dwuwymiarowych ograniczonych izolatorów atomowych na bazie pętli Wilsona i wskaźników symetrii. Okazuje się, że monowarstwa antymonu, pośród innych przedstawionych potencjalnych materiałów, jest w rzeczywistości takim ograniczonym izolatorem i wykazuje chronione symetrią ładunki zlokalizowane na rogach sieci.

W trzeciej części omawiane jest wzajemne oddziaływanie topologii i niehermitowskości w hamiltonianach opisujących efektywnie układy otwarte. Wprowadzenie członu niehermitowskiego prowadzi do wyjątkowych zjawisk takich jak punkty wyjątkowe (ang. exceptional points) oraz anomalna lokalizacja wszystkich stanów własnych na brzegu zwana efektem naskórkowości (ang. skin effect). Posługując się modelem $\pi$-flux na sieci kwadratowej z dodatkowym członem niehermitowskim, został zademonstrowany nowy efekt nazwany \emph{odwrotnym} efektem naskórkowości, który nie wymaga żadnych całek przeskoku zależnych od kierunku. Przewidywania teoretyczne zostały poparte wynikami eksperymentalnymi uzyskanymi poprzez pomiary specjalnie zaprojektowanego obwodu elektrycznego, który realizuje fizykę niehermitowskiego modelu $\pi$-flux.

\vspace{1cm}
\hspace{-0.4cm}\textbf{Słowa kluczowe:} fazy topologiczne, modele ciasnego wiązania, fraktale, model Hofstadtera, splątanie kwantowe, (krystaliczne) izolatory topologiczne, hamiltoniany niehermitowskie
\end{otherlanguage}

\endgroup

\vfill