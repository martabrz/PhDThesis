\cleardoublepage\phantomsection
\chapter*{Introduction}
\addcontentsline{toc}{chapter}{%
  \texorpdfstring{\spacedlowsmallcaps{Introduction}}{Introduction}%
}
\label{ch:introduction}


Classification is a theme that lies at the heart of condensed matter physics. For a long time, it was believed that Ginzburg-Landau theory~\cite{Landau1950} of the phase transitions based on the symmetry-breaking paradigm provides a complete list of phases of matter. It states that the continuous transition between phases is described by a local order parameter, which vanishes in the high-symmetry (disordered) phase and becomes non-zero in the low-symmetry (ordered) phase. For instance, magnetization serves as an order parameter at the ferromagnetic-paramagnetic transition; crystals break continuous translational symmetry and hence are characterized by a discrete space group. Another example is superconductivity, where the non-zero gap $\Delta$ indicates the superconducting state.

At the beginning of 1970s, Berezinskii~\cite{Berezinsky:1970fr, Berezinsky:1972rfj}, Kosterlitz and Thouless~\cite{Kosterlitz_1972, Kosterlitz_1973} investigated a two-dimensional classical magnet with $U(1)$ (XY model) in which the phase transition falls beyond the Ginzburg-Landau paradigm. At the zero-temperature, the system is ferromagnetic as the spins are perfectly arranged. At finite (but small) temperature, the system still has long-range correlations but a local spin structures called vortices may be created. Vortices are said to be topological excitations as the only way to destroy them is to annhilate vortex with an antivortex. With an increase of the temperature, more pairs of vortex-antivortex pairs are created and they become less bounded, hence destroying the long-range correlations. This is the Kosterlitz-Thouless (KT) phase transition in which no continuous symmetry is spontaneously broken (which is in an agreement with Mermin-Wagner theorem), yet the system undergoes a transition to the phase with short-range correlations. For their developments in the field of topological matter, Kosterlitz and Thouless (together with Haldane) obtained the Nobel Prize in 2016.

An even more pronounced example is a discovery of the integer quantum Hall effect (IQHE)~\cite{IQHE1980}, and subsequently, the fractional version of this phenomenon (FQHE)~\cite{FQHE1982}. In a two-dimensional electron gas at low temperature exposed to a strong magnetic field, applying the voltage on the two sides of a sample results in a current generated in the perpendicular direction. As a function of the magnetic field, one observes the perfectly flat plateaus in the transverse Hall conductivity, while the longitudinal conductivity vanishes. Hall conductivity $\sigma_{xy}$ takes quantized values being multiplies $\nu$ ($\nu$ is an integer in case of IQHE, while a fraction in case of FQHE) of elementary constants $e^2 / h$ (where $e$ is the electron charge and $h$ is the Planck constant) and has been measured to the accuracy of the order $10^{-9}$~\cite{codata2016}. For the discovery of IQHE, Klaus von Klitzing got the Nobel Prize in 1985 and now the quantum of conductance serves as a universal constant. The origin of this quantization is universal, in a sense that it is observed regardless of microscopic details of a sample such as disorder. In a finite geometry, the systems exhibits robust edge currents, which are chiral, that is they flow in one fixed direction. These are two different manifestations of the topological properties of the systems.

IQH states can be understood from the perspective of the Landau levels formed in a strong magnetic field. A remarkable idea given by Thouless, Kohmoto, Nightingale, and den Nijs (TKNN)~\cite{TKNN1982} was to relate the number of gapless edge modes with the topological invariant computed for the bulk. This became the first observation of the  \emph{bulk-boundary correspondence}. Later on, the theoretical proposals of realization of IQH states on the honeycomb lattice in the absence of magnetic field~\cite{Haldane1988} or topological superconductors~\cite{TSCRead2000} ignited the experimental search for topological materials. On the other hand, understanding FQHE requires taking into account the electron-electron interactions. The exotic properties emerge from the collective behavior of electrons and give rise to the concept of topological order as the low-energy effective theory can be described in terms of topological quantum field theories (such as Chern-Simons or BF). Apart from FQHE, another paradigmatic example of topological order is chiral spin liquid developed as an attempt to understand high-temperature superconductivity~\cite{CSLWen1989, CSSWen1989}. Topologically ordered states may posses unique features such as fractionalized excitations (carrying the fractions of elemental charge) with the anyonic exchange statistic (not bosonic nor fermionic) or long-range entanglement pattern~\cite{RevWen2017}. In addition, they exhibit robust ground state degeneracy depending on the manifold on which they are defined. These properties are very promising for a fault-tolerant quantum computing~\cite{RevNayak2008}.

As no symmetry breaking occurs, in order to classify topological phases a notion of equivalence classes has to be introduced. To do so, we will investigate the systems at zero temperature with a spectral gap separating the ground state from the first excited state. States are said to be topologically equivalent if they can be connected by unitary transformations that act infinitely slowly (adiabatically). The physical constrains of unitaries are that at every point of evolution they have to preserve the energy gap and they have to involve only local degrees of freedom. In addition, one may take into account the symmetries to enlarge the number of possible unitary transformations and investigate symmetry-protected (or symmetry-enriched) topological states.

Not all the states that are topological posses intrinsic topological order: some of them may be short-range entangled and they can be understood from the single-particle physics perspective (a more detailed explanation will be given in the following chapter). Interestingly, a lot of theoretical developments were followed-up by experimental efforts.  

\section*{Organization of this thesis}
In this thesis, we study non-interacting fermionic systems which cannot be completely classified by means of the so-called ten-fold way, that is the classification based on a dimensionality and the presence or absence of internal symmetries: time-reversal, particle-hole and chiral.

Chapter~\ref{ch:topo-intro} serves as a brief introduction to topological band theory, with an emphasis on the basic definitions, topological invariants and illustrative examples of toy models. 

In Chapter~\ref{ch:fractals}, a realization of topological states in systems with non-integer spatial dimension is discussed. Detecting topological properties without the translational invariants requires employing real-space methods

Chapter~\ref{ch:oals} is devoted to the states protected by the crystalline symmetries. However, they fall beyond a sharp distinction what is topological or trivial. Obstructed atomic limits are recent refinements and can be understood from the perspective of Wannier functions.

Topological states in non-Hermitian systems are investigated in Chapter~\ref{ch:nh}. Recently, a studies of non-Hermitian Hamiltonians emerged as an effective modeling of open systems in which the energy or particle number is not preserved.

Finally, Chapter~\ref{ch:summary} provides a summary and indicates further directions. 

In addition, some lengthly derivations are given in Appendix~\ref{ch:appendix}.