\chapter{Introduction}
\label{ch:introduction}
In recent years, we have witnessed impressive developments in the field of condensed matter physics, both from theoretical and experimental perspectives. In particular, the discoveries of new phases of matter beyond the Landau paradigm - starting from the Integer Quantum Hall effect and, subsequently, its fractional version, through unconventional superconductors and more, have stimulated many people to find broader and more general classifications schemes. 

To understand the physics of QH systems, back in the 80s people tried to employ the notion of topology as a way to understand some peculiar physical properties that cannot be captured by the local order parameters. Later on, the seminal paper by Haldane showed the concept of QH systems without the Landau levels - what is called now the Chern insulator. A reignited interest in topological aspects of quantum states appeared in 2005 with the papers by Kane and Mele (QSH in graphene) as well as Bernevig-Hughes-Zhang (QW in HgTe) .


In 2008, the idea of classification based on the spatial dimensionality and the presence (or absence) of internal symmetries - time-reversal, particle-hole and chiral, was provided using different methods (K-theory, ...), the so-called ten-fold way. Later attempts were trying to capture the relevance of crystal symmetries in the protection of topological states. Also non-Hermiticity extends this classification \todo{bla bla}.


In this thesis, we would like to present and discuss novel topological states which fall beyond the ten-fold way. To familarize the reader with basic concepts, we start with a short introduction to topological band theory in Chapter~\ref{ch:topo-intro}. In Chapter \ref{ch:fractals} we discuss the IQHE on a class of self-similar lattices. Chapter \ref{ch:oals} is devoted to the phases of intermediate stability, which are not defined by strong topological indices. Non-Hermitian physics will be covered in Chapter \ref{ch:nh}. Finally, the summary and further remarks are given in Chapter \ref{ch:summary}.