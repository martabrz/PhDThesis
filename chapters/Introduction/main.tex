\cleardoublepage\phantomsection
\chapter*{Introduction}
\addcontentsline{toc}{chapter}{%
  \texorpdfstring{\spacedlowsmallcaps{Introduction}}{Introduction}%
}
\label{ch:introduction}
The classification of the different states of matter and the transition between them is a theme that lies at the heart of condensed matter physics. For a long time it was believed that Ginzburg-Landau theory~\cite{Landau1950}, based on the spontaneous symmetry breaking paradigm, provides a complete description of phase transitions. It states that continuous transitions between phases can be described by a local physical observable -- the order parameter -- which vanishes in the high-symmetry (disordered) phase and becomes non-zero in the low-symmetry (ordered) phase. In magnetic systems, for instance, a magnetization serves as the order parameter at the ferromagnetic-paramagnetic transition. Crystals, on the other hand, break the continuous translational symmetry, leading to a characterization in terms of space groups. Superconductivity is another example, where the $\mathrm{U}(1)$-gauge symmetry is broken in the superconducting state.

However, Berezinskii~\cite{Berezinsky:1970fr, Berezinsky:1972rfj}, in parallel with Kosterlitz and Thouless~\cite{Kosterlitz_1972, Kosterlitz_1973}, pointed out that the Ginzburg-Landau description is not exhaustive. They investigated a two-dimensional classical magnet with $\mathrm{U}(1)$ symmetry, the XY model, in which the phase transition is not accompanied by the breaking of any symmetry. The model describes interacting planar spins. At zero-temperature, the system is ferromagnetic as the spins are perfectly arranged. At finite but small temperatures, the system still has long-range correlations, but local spin structures called vortices may be created. Vortices are said to be topological excitations as the only way to destroy them is to annihilate a vortex with another vortex that rotates in an opposite direction, an antivortex. With increasing temperature, more vortex-antivortex pairs are created. Above a critical temperature, the pairs are no longer bounded together, and as a consequence, long-range correlations are destroyed. This constitutes the Kosterlitz-Thouless (KT) phase transition, where the system undergoes a transition to a phase with short-range correlations without spontaneous symmetry breaking\footnote{The KT phase transition does \emph{not} violate the Mermin-Wagner theorem because the transition from a phase with power-law correlations to a phase with exponentially decaying correlations occurs via the unbinding vortex-antivortex pairs.}. In particular, the KT transition exemplified the importance of the abstract language of topology in the context of phase transitions. 

The notion of topology became even more salient with the discovery of the quantum Hall effect in its integer (IQHE)~\cite{IQHE1980} and, subsequently, fractional (FQHE) version~\cite{FQHE1982}. When an electron gas confined in two dimensions and at low temperature is exposed to a strong magnetic field, the transverse Hall conductivity exhibits perfectly flat plateaus, while the longitudinal conductivity vanishes. In the bulk, the electronic states form Landau levels with a finite gap between them. At each plateau, the Hall conductivity $\sigma_{xy}$ is a multiple integer of the quantum conductance $e^2/h$, $\sigma_{xy} = \nu \, (e^2 / h)$, where $\nu$ is the electronic filling fraction. $\nu$ is either an exact integer (IQHE) or a well-defined fraction (FQHE), and has been experimentally measured to an astonishing precision of $10^{-9}$~\cite{codata2016}. This quantization is universal, \ie, it is observed regardless of the microscopic details of the sample such as disorder. In an open geometry, the bulk material remains insulating (if a chemical potential lies within a gap), but the current flows in one fixed direction along the physical edge. In fact, the quantized $\sigma_{xy}$ and the existence of robust chiral edge modes are distinct manifestations of the topological nature of the system. Thouless, Kohmoto, Nightingale, and den Nijs~\cite{TKNN1982} linked the Hall conductivity in IQH systems to a topological invariant called the (first) Chern number. The Chern number was found to be also related to the number of gapless states at the boundaries of the system~\cite{PhysRevLett.71.3697}, justifying the celebrated bulk-boundary correspondence. The FQHE, on the other hand, cannot be explained through a simple single-particle, non-interacting picture. The FQHE is a paradigmatic example of topological order where exotic properties emerge from the collective behavior of electrons. The presence of electron-electron interactions may give rise to, for example, low-energy fractionalized excitations (carrying fractions of the elemental charge) with anyonic exchange statistic (neither bosonic nor fermionic) or long-range entanglement pattern characterized by a non-vanishing topological entanglement entropy~\cite{RevWen2017}. In addition, topologically ordered phases exhibit a robust ground state degeneracy depending on the manifold on which they are defined. These properties are very promising ingredients for devising a fault-tolerant quantum computer~\cite{RevNayak2008}.

In 1988, Haldane~\cite{Haldane1988} showed that it is possible to realize quantum Hall states without a net magnetic field, now known as Chern insulators. But a breakthrough in the field of topological matter occurred almost 20 years later with a proposal by Kane and Mele~\cite{KaneMeleGraphene, PhysRevLett.95.146802}. The authors studied the effect of a strong spin-orbit coupling on electronic structure of graphene and discovered that the constraints imposed by time-reversal symmetry could lead to a novel gapped state of matter with metallic edge states. The quantum spin Hall (QSH) insulating phase, in contrast to the quantum Hall effect, is solely protected by time-reversal symmetry and described by a different topological invariant called the $\mathbb{Z}_2$ index. Soon after, a realization of QSH in HgTe/CdTe quantum wells was suggested theoretically~\cite{Bernevig1757} and confirmed experimentally~\cite{MolenkampQSHE2007}, where the key mechanism to obtain non-trivial bulk topology is a band inversion. The concept of $\mathbb{Z}_2$ topological states carries on to three dimensions~\cite{PhysRevLett.98.106803, PhysRevB.75.121306, PhysRevB.79.195322} with conducting surfaces as characteristic features, which were firstly observed in bismuth-based materials~\cite{Hsieh2008, Zhang2009, Xia2009}.

One of the milestones in the field was the construction of the topological classification of gapped, free-fermionic systems by Ryu, Schnyder, Furusaki and Ludwig~\cite{10foldSchnyder2008, 10foldRyu2010}  and independently by Kitaev~\cite{10foldKitaev2009}. Any generic quadratic Hamiltonian can be assigned to one of ten symmetry classes (as noticed by Altland and Zirnbauer~\cite{AltlandZirnbauer1997}) based on its spatial dimensionality and the presence (or absence) of time-reversal, particle-hole and chiral symmetries and labeled by a relevant topological invariant. The so-called symmetry-protected topological (SPT)\footnote{Note that this abbreviation also stands for the symmetry-protected topological order (or symmetry-enriched topological, SET) described by a Hamiltonian with a spectral many-body gap and equipped with certain symmetries. These phases are beyond the scope of the thesis.} phases cannot be adiabatically connected to a trivial insulating limit (\ie product state) without closing the bulk gap or breaking the relevant symmetries. This includes topological insulators (TIs) and topological superconductors (TSCs) with internal symmetries~\cite{RevModPhys.83.1057}. Although the ten-fold way is regarded as a well-established classification scheme, it is clearly not exhaustive. 

Given the recent developments in the context of topological phases of matter, the topological classification based on ten symmetry classes can be extended in several ways. For instance, the topological classification can carry on to bosonic systems~\cite{zhou2019classification} or even to classical systems~\cite{SusstrunkE4767}. The notion of SPT phases has also been extended to gapless phases, such as topological semimetals~\cite{TSMrev} revealing the linear dispersion around the nodes and unconventional superconductors with point (or line) nodes~\cite{Schnyder_NODAL}. Topological metals and semimetals can be categorized by the dimensionality of the topologically protected band degeneracies~\cite{PhysRevB.88.125129, PhysRevB.90.205136,RevModPhys.88.035005}. Periodic, time-dependent Hamiltonians are also the subject of generalization of the ten-fold way~\cite{PhysRevB.96.155118}.

The interplay between topology and symmetries is still an exceptionally active research area. Spatial symmetries, which are ubiquitous in crystals, enrich the list of possible gapped and gapless topological phases~\cite{PhysRevB.90.165114, PhysRevB.88.125129, PhysRevB.90.205136, PhysRevB.99.075105} with topological crystalline insulators (TCIs)~\cite{tcirev} and higher-order topological insulators (HOTIs)~\cite{Schindler2018}. Open dissipative systems described by non-Hermitian Hamiltonians can be systematized in a same manner as Hermitian systems~\cite{PhysRevB.101.205417, PhysRevX.9.041015, PhysRevLett.120.146402}. The topological properties of all the aforementioned systems can be explained in terms of non-interacting or mean-field Hamiltonians. Nevertheless, a general unifying framework has not been yet constructed.

\section*{Outline}
In this thesis, we study selected free-fermionic topological systems whose properties cannot be deduced from the ten-fold classification. The topics we will explore can be thought as three research directions in which the existing classification remains incomplete. Firstly, in Chapter~\ref{ch:fractals}, we present the systems which are defined in \emph{non-integer spatial dimensions}. Then, in Chapters~\ref{ch:tci} and~\ref{ch:oals}, we discuss the importance of \emph{spatial symmetries} on the realization of symmetry-protected topological phases and also cast light on the effect of topology in the presence of crystalline symmetries. Finally, in Chapter~\ref{ch:nh}, we explain how \emph{non-Hermiticity}, originating from the effective description of open systems, may profoundly affect topological properties and enrich the classification. The thesis is organized as follows:

\vspace{0.22cm}
\noindent Chapter~\ref{ch:topo-intro} serves as an introduction to topological band theory. We start with key definitions related to band topology, the role of internal symmetries and the classification with respect to them, together with relevant topological invariants. We discuss in more details the properties of the ten-fold way and give an overview of the bulk-boundary correspondence. For illustration purposes, we provide several examples of models in different symmetry classes.

\vspace{0.22cm}
\noindent In Chapter~\ref{ch:fractals}, we investigate whether general graphs --  in particular fractals -- can host topological states by the means of case studies. Here, we consider the Hofstadter model defined on two fractal lattices: the Sierpiński carpet and gasket, and combine several numerical techniques to characterize their topological properties. Starting from the analysis of numerically obtained energy spectra and eigenstates, we then move to direct calculations of topological invariants. Due to the lack of translational invariance, we compute the Bott index and the Chern number represented in real space. In addition, we examine the effect of on-site disorder through the level spacings statistics and the Chern number. We also compare the results with regular lattices. The content of this Chapter is based on the article \emph{Topology in the Sierpiński-Hofstadter problem}.
 
\vspace{0.22cm}
\noindent Chapter~\ref{ch:tci} includes studies on topological properties of atomically thin bismuth and antimony layers described within a tight-binding formalism. Their completely flat counterparts -- bismuthene and antimonene -- exhibit a topological crystalline insulating phase protected by mirror symmetry, which we confirm by computing the mirror Chern number. We show that the entanglement entropy and entanglement spectrum, conveniently expressed in terms of the eigenvalues of the two-point correlation matrix, can be used as complementary tools for defining bulk topology. Additionally, we investigate topological phase transitions induced by doping, external electric field and strain. The effect of experimentally relevant silicon carbide (SiC) substrate on the monolayers is also discussed. The content of this Chapter is based on the articles \emph{Entanglement entropy and entanglement spectrum of Bi$_{1-x}$ Sb$_{x}$ (111) bilayers} and \emph{Topological phases in Bi/Sb planar and buckled honeycomb monolayers}.

\vspace{0.22cm}
\noindent Chapter~\ref{ch:oals} is devoted to obstructed atomic limits, a novel class of insulating states stemming from the crystalline symmetries. We extend the classification of these phases in two dimensions to the spinful case with time-reversal symmetry by employing (nested) Wilson loop invariants and symmetry indicators. To predict the presence of quantized, symmetry-protected corner charges -- the boundary manifestations of non-trivial bulk topology -- we provide useful formulas based on Bloch wavefunctions. Our theoretical findings are supported by density functional theory results for arsenic and antimony monolayers in an open flake geometry. The content of this Chapter is based on the article \emph{Fractional corner charges in spin-orbit coupled crystals}.

\vspace{0.22cm}
\noindent The interplay between topology and non-Hermiticity is investigated in Chapter~\ref{ch:nh}. We discuss the phenomena intimately related to the non-Hermiticity: the presence of degeneracies in the complex spectrum called exceptional points and the breakdown of bulk-boundary correspondence associated with the anomalous localization of the eigenstates named the skin effect. To demonstrate these effect, we examine the $\pi$-flux model on a square lattice with additional non-Hermitian hopping terms. Based on results obtained for a quantum lattice model, we design, describe, and measure a non-Hermitian topolectrical circuit realizing a new, reciprocal type of the skin effect. The content of this Chapter is based on the article \emph{Reciprocal skin effect and its realization in a topolectrical circuit}.

\vspace{0.22cm}
\noindent Finally, Chapter~\ref{ch:summary} summarizes the results and indicates potential research directions. Additional definitions and derivations are given in Appendices at the end of the main text.