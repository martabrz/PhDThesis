\chapter{Conclusions and outlook}
\label{ch:summary}
 
The immense interest in the field of topological matter over the last years, both from theoretical and experimental perspectives, has significantly extended the number of known topological phases. In this thesis, we have explored selected low-dimensional non-interacting fermionic phases which fall outside the classification of topological insulators and superconductors based on internal symmetries -- the ten-fold way. Below, we summarize the original results from Chapters~\ref{ch:fractals}--~\ref{ch:nh} and suggest potential research directions.

\section*{Chapter 2}
\noindent In Chapter~\ref{ch:fractals}, we have discussed possible realization of topological phases in fractal geometries with a non-integer Hausdorff dimension. By means of a case study, we investigated the Hofstadter model defined on the Sierpiński carpet and gasket. We showed that both systems exhibit a hierarchy of the edge-localized states at different levels of fractals depth in a finite magnetic field. Using the Bott index and the real-space formulation of the Chern number, we distinguished topologically non-trivial regions in the spectrum and probed them by performing the level statistics analysis in the presence of disorder. Therefore, a key finding is that characteristic features of quantum Hall systems in two dimensions are also observed in \emph{almost} two-dimensional lattice models.

It is imperative to ask what dimensional and connectivity properties a graph must have in order to support topological states. Our results call for an extension of the ten-fold way to more general graphs and question the bulk-boundary correspondence in systems without a sharp distinction between the bulk and the edge. Also, the realization of topologically ordered phases in fractal geometries still remains essentially uncharted (only recently, Ref.~\cite{manna2019anyons} provided a method to construct FQH states on fractal lattices hosting anyons). An interesting research direction would be to provide a systematic classification for scale-invariant systems and define topological invariants in analogy to the ones based on occupied Bloch states. While the Chern number definitely brought valuable insight into topological states on fractals, it is an open question whether it is truly the appropriate invariant in systems with fractional Hausdorff dimension. Consequently, more careful studies should be carried out for fractals with an integer Hausdorff dimension, but embedded in a higher-dimensional space such as Sierpiński tetrahedron -- a three-dimensional object, but characterized by a $d_H$ equal to 2.

\section*{Chapter 3}
\noindent Going further, in Chapter~\ref{ch:tci} we discussed the role of crystal symmetries in symmetry-protected topological phases. We focused on topological crystalline insulating states protected by the mirror symmetry, with the concrete examples of bismuth and antimony monolayers due to their highly-tunable electronic properties. Using a tight-binding model, we showed how to drive a system to a topological phase, but also how to induce a topological phase transition between topologically distinct TI and TCI phases. In addition, we demonstrated that single-particle entanglement measures can provide supplemental information on the topological properties of systems compared to standard band structure analysis, even for small system sizes.

Composition- and strain-induced phase transitions reveal a finite discontinuity in the entanglement entropy. On the other hand, the electric field-driven topological phase transition seems to have a qualitatively different character as the entanglement entropy remains a continuous function with respect to the electric field strength, while its first derivative is discontinuous. We relate this difference to the breaking of inversion symmetry in the latter case, however additional checks should be carried out. A long-standing belief was that a TPT between trivial and non-trivial phases is always of the second order as the band gap continuously changes with respect to model parameters. Including electron-electron interactions may result in a first-order transition~\cite{PhysRevLett.114.185701, PhysRevB.94.041101, PhysRevB.94.035109}, but even more interesting is that such a transition may happen at the level of non-interacting systems~\cite{PhysRevB.95.161403}, signatures of which were most likely observed in Sn-doped PbSe~\cite{PhysRevB.90.161202} in the TCI phase. We speculate that careful studies of the scaling of EE across the critical point would allow to characterize the order of TPTs.

In this Chapter, we considered only one type of spatial bipartition and kept the length between the subsystems $\ell$ constant, but systematic studies of the scaling of entanglement entropy depending on the size and shape of the spatial cut may be insightful. For instance, observing the area law around a critical point by changing $\ell$ would allow to extract the central charge and, ultimately, find the universality class of the second-order phase transition. General arguments may be formulated by investigating the sign of the correlation function constructed from Wannier functions and the corresponding correlation length close to criticality~\cite{PhysRevB.95.075116}.

\section*{Chapter 4}
\noindent As established in Refs.~\cite{Miert16,miertOrtixFractionalCharge17,rhim17} for insulators with bulk polarization and more recently in Refs.~\cite{miertcorners,benalcazar2018quantization} for second-order topological insulators, the non-trivial bulk topology of OALs can be revealed via the charge fractionalization at their boundaries. This represents the simplest mechanism for a topological bulk-boundary correspondence that is protected by crystalline symmetries. In Chapter~\ref{ch:oals}, we presented theory and material candidates for charge fractionalization at corners in 2D systems with significant spin-orbit coupling, thus providing a broader picture than the one presented in some previous treatments of this phenomenon~\cite{miertcorners,EzawaWannier19,benalcazar2018quantization}. Corner charges in topological insulators are well-defined when there is no edge spectral flow but also only in the absence of an edge-induced filling anomaly due to (time-reversal) bulk polarization. Since there is no crystalline symmetry-protected edge spectral flow in 2D (assuming the symmetry acts at least in part non-locally), corner charges are well-defined for all 2D systems that are not strong or weak first-order topological insulators, or $M_z$ mirror Chern insulators~\cite{FuTCI2011,tcirev}.

Diagnosing spinful OALs with time-reversal symmetry in 2D was particularly challenging because the irreducible representations of the occupied bands at HSPs are usually two-dimensional, yielding trivial symmetry indicator invariants at $C_2$-invariant HSPs. Symmetry indicators were therefore insufficient to identify the Wannier centers in $C_6$, $C_4$, and $C_2$ symmetric insulators. To overcome this difficulty, we considered Wilson loop and nested Wilson loop invariants, which could better 'resolve' the positions of the Wannier centers. Wilson loops, however, are essentially one dimensional objects that extract projections of the 2D positions of the Wannier centers along particular directions. Nested Wilson loops are a best-effort attempt to localize the Wannier centers in 2D, but cannot always be interpreted literally due to the possible non-commutation of Wilson loops along different directions. In the presence of crystalline symmetries, however, Wilson and nested Wilson loops have eigenvalues with quantized phases, which clearly distinguish different OALs in $C_6$ and $C_2$ symmetric insulators, but are insufficient for insulators which only have $C_4$ symmetry. Therefore, finding a formula for the corner charge in such $C_4$-symmetric systems would be a possible future research direction.

We studied the protection due to spatial symmetries only because corner charge fractionalization is a robust observable that does not require additional spectral symmetries such as chiral or particle-hole symmetry. However, when particle-hole symmetry is present, we can additionally predict topologically protected zero-energy corner states. These are characterized by the charge $Q_\mathrm{c} = 1 \mod 2$: Consider a system with an $n$-fold symmetry in a phase with $2n$ degenerate midgap states (the 2 is due to TRS). At half-filling, $n$ midgap states are occupied and there is no gap. To arrive at a gapped system (as required for the corner charge to be well defined), we need to either fill $n$ more states or remove $n$ electrons from the charge-neutral system. When maintaining the crystal symmetry, this implies an excess (or missing) charge of $Q_\mathrm{c} = 1 \mod 2$ for each of the $n$ corners~\cite{benalcazar2018quantization}.

Interestingly, we find that there are obstructed atomic limits, where the electrons are localized away from the atomic sites, which still do not have non-trivial corner charges. These may instead be diagnosed by their response to crystal defects~\cite{benalcazar2018quantization,li2019fractional}. Topological defects such as dislocations have been proven to be useful in probing the bulk properties of conventional TCIs~\cite{geier2020bulkboundarydefect}, HOTIs~\cite{PhysRevLett.123.266802, roy2020dislocation} and boundary-obstructed phases~\cite{tiwari2020chiral}. We leave the exploration of the defect response of obstructed atomic limits with significant spin-orbit coupling to future work. 

The developed formulas for the corner charges, in particular based on symmetry indicators, may be applicable in high-throughput calculations for materials discovery. So far, large-scale computations have been performed for strong topological~\cite{Vergniory2019, Tangeaau8725, Tang2019, PhysRevMaterials.3.024005} and fragile~\cite{MaiaFragile3} phases. Hence, it would be of a great interest to systematically study OALs in a similar way.

\section*{Chapter 5}
\noindent Finally, in Chapter~\ref{ch:nh}, we introduced and experimentally demonstrated the concept of the reciprocal skin effect, where the breakdown of bulk-boundary correspondence occurs in the absence of any non-reciprocal coupling. Instead of having extensive mode accumulation all along one boundary, an equal number of eigenmodes localizes along opposite boundaries, with the direction of localization tied to the momentum component parallel to the boundary. Key to their realization is the gain/loss associated with couplings across different sites, which effectively behave like non-reciprocal couplings at a fixed transverse momentum. The reciprocal skin effect can in principle exist in two or higher dimensions when the non-Hermitian reciprocal couplings connect different internal degrees of freedom and the momentum space structure protects the skin modes from hybridization.

We observed the reciprocal skin effect in an electric circuit with solely passive linear circuit elements. The breakdown of bulk-boundary correspondence becomes evident by comparing the PBC system with exceptional points with the markedly different OBC case, featuring oppositely localized skin modes. Our circuit, and more generally the reciprocal skin effect, facilitates novel functionalities when coupled to electromagnetic waves. For instance, it lends itself to potential applications for polarization and direction detectors for electromagnetic waves, where differently directed or polarized input signals are substantially accumulated towards opposite directions.

A slightly different question to the problem investigated here is whether it is possible to \emph{suppress} the skin effect intrinsically present in a system. In Ref.~\cite{PhysRevB.101.121109}, the authors proposed to consider a many-body wave function, which includes the fermionic statistics and therefore prevents the states to pile up at the edges due to the Pauli exclusion principle. A natural way to preclude the anomalous localization at the boundary would be to introduce disorder into the system, which is a subject of recent studies~\cite{PhysRevB.100.054301, claes2020skin}. We also speculate that incorporating crystal symmetries in the discussed non-Hermitian $\pi$-flux model may prevent it from forming the skin effect. Only recently, the concept of higher-order skin effect has been introduced~\cite{kawabata2020higherorder, fu2020nonhermitian, okugawa2020secondorder}. In analogy to higher-order TIs, the skin states can localize in the corners of 2D systems and on the hinges in 3D in the presence of spatial symmetries.

Another not fully explored direction is the interplay between non-Hermiticity and interactions. Several non-Hermitian extensions to the interacting problems were proposed, for example Kondo lattice models~\cite{PhysRevB.98.035141, PhysRevLett.121.203001} or FQH states~\cite{Yoshida2019}. Conceivably, there may be a way to construct novel non-Hermitian topologically ordered states without Hermitian equivalent. In general, non-Hermitian systems are challenging to study from the numerical perspective as ill-conditioned matrices often give rise to numerical instabilities. Therefore, it would be highly relevant to adapt well-developed numerical techniques for the many-body problems such as tensor networks to efficiently describe non-Hermitian states.