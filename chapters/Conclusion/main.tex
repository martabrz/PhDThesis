\cleardoublepage\phantomsection
\chapter*{Summary}
\addcontentsline{toc}{chapter}{%
  \texorpdfstring{\spacedlowsmallcaps{Summary}}{Summary}%
}
\label{ch:summary}
Recent years have witnessed an immense interest in finding and exploring topological materials, both from theoretical and experimental perspectives, as they hold great promise for technological applications ranging from the electronics to quantum computing. In this thesis, we have studied three different classes of topological phases that fall beyond the well-established classification of topological insulators and superconductors.

In Chapter~\ref{ch:fractals}, we have discussed the realization of topological phases in fractal geometries with a non-integer Hausdorff dimension. By means of a case study, we investigated the properties of Sierpiński carpet and gasket in a homogeneous magnetic field. A key finding is that characteristic features of quantum Hall systems in two dimensions are also observed in \emph{almost} two-dimensional lattice models. Therefore, it is imperative to ask what dimensional and connectivity properties a graph must have in order to support topological states. Obtained results call for an extension of the ten-fold way to more general graphs and impose a question about the bulk-boundary correspondence in systems without a sharp distinction between the bulk and the edge. An interesting research direction would be to provide a systematic classification for scale-invariant systems and define topological invariants in analogy to Bloch-type invariants for models with translational invariance.

In Chapter~\ref{ch:oals}, a notion of obstructed atomic limits has been introduced. To support our theoretical findings, we proposed monolayers of bismuth, antimony and arsenic as material candidates. High-throughput calculations combining density functional theory and machine learning algorithms have been already performed, but mostly for bulk materials. Hence, it would a useful (and challenging) task to carry out large-scale computations for two-dimensional OALs.

Finally, in Chapter~\ref{ch:nh} we have studied models defined by non-Hermitian Hamiltonians. In particular, we demonstrated experimentally the \emph{reciprocal} skin effect using a topolectrical platform. Despite rapid development in this field, there are several not fully explored directions. One of them is the interplay between non-Hermiticity and interactions. Conceivably, there may be a way to construct non-Hermitian topologically ordered states without Hermitian counterparts.
